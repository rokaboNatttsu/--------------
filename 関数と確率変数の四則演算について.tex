\documentclass[a4paper,11pt]{ltjsarticle}
\usepackage{amsmath}
\begin{document}
\tableofcontents
\newpage



\section{共通する前提}
内容の正しさは自信ない。特に数学記号の使い方。\\

これ以降、共通して用いる変数・関数\\
$f(x), g(y), h(z)$:確率密度関数\\
$X, Y, Z$:確率変数\\
$A$:(確率変数ではない)変数および定数\\
$i,j,k,N,M,L$:自然数\\
$X=f(x), Y=g(y), Z=h(z)$\\
特に断りがなければ、X,Y,Zは独立な確率変数\\
特に断りがなければ、x,y,zは独立な変数
\newpage


\section{確率変数Xと定数Aの四則演算}
\subsection{和・足し算}
$Y=X+A$のとき。$g(y) = f(x-A)$\ \ \ \ ($y=x-A$)
\subsection{差・引き算}
\subsubsection{X-A}
$Y=X-A$のとき。$g(y) = f(x+A)$\ \ \ \ ($y=x+A$)
\subsubsection{A-X}
$Y=A-X$のとき。$g(y) = f(-x-A)$\ \ \ \ ($y=-x-A$)
\subsection{積・掛け算}
$Y=AX$のとき。$g(y) = \frac{f(x/A)}{\vert A\vert }$\ \ \ \ ($y=x/A$)
\subsection{商・割り算}
\subsubsection{X/A}
$Y=X/A$のとき。$g(y) = \vert A\vert f(Ax)$\ \ \ \ ($y=AX$)\\
Aで割るのではなくAの逆数をかけると考えることを推奨。
\subsubsection{A/x}
$Y=A/X$のとき。$g(y) = ???$\ \ \ \ ()\\
この章、何か重大な間違いをしているような気がする。
\newpage


\section{確率密度関数が連続値の確率変数X,Yの四則演算}
\subsection{Z=X+Y}
足し算だから、x,y,zの単位は共通でなければならないことに留意。
\subsubsection{不定積分と-∞から∞までの定積分}
不定積分
$$
    h(z) = \int f(z-y) g(y)\,dy
$$

-∞から∞までの定積分
$$
    h(z) = \int_{-\infty}^{\infty} f(z-y) g(y)\,dy
$$
\subsubsection{定積分}
\begin{description}
    \item xの積分区間は$x_0 \leq x \leq x_1$
    \item yの積分区間は$y_0 \leq y \leq y_1$
\end{description}
$$
    h(z)=
    \begin{cases}
        \int_{y_0}^{y_1} f(z-y) g(y)\,dy    & if\ x_0+y \leq z \leq x_1+y \ \ \forall y\in \{ y_0 \leq y \leq y_1\}, \\
        積分区間を分割するべし                & else \ if\ x_0+y \leq z \leq x_1+y \ \ \exists y\in \{ y_0 \leq y \leq y_1\}, \\
        0                                   & otherwise,
    \end{cases}
$$
(一番上の条件、$y_0 \leq y \leq y_1$を満たすすべての$y$が$x_0+y \leq z \leq x_1+y$を満たす場合。と言いたい)


\subsection{Z=X-Y}
引き算だから、x,y,zの単位は共通でなければならないことに留意。
\subsubsection{不定積分と-∞から∞までの定積分}
不定積分
$$
    h(z) = \int f(z+y) g(y)\,dy
$$

-∞から∞までの定積分
$$
    h(z) = \int_{-\infty}^{\infty} f(z+y) g(y)\,dy
$$
\subsubsection{定積分}
\begin{description}
    \item xの積分区間は$x_0 \leq x \leq x_1$
    \item yの積分区間は$y_0 \leq y \leq y_1$
\end{description}
$$
    h(z)=
    \begin{cases}
        \int_{y_0}^{y_1} f(z+y) g(y)\,dy    & if\ x_0-y \leq z \leq x_1-y \ \ \forall y\in \{ y_0 \leq y \leq y_1\},\\
        積分区間を分割するべし                & else \ if\ x_0-y \leq z \leq x_1-y \ \ \exists y\in \{ y_0 \leq y \leq y_1\}, \\
        0                                   & otherwise,
    \end{cases}
$$
(一番上の条件、$y_0 \leq y \leq y_1$を満たすすべての$y$が$x_0-y \leq z \leq x_1-y$を満たす場合。と言いたい)


\subsection{Z=XY}
積の計算。
\subsubsection{不定積分と-∞から∞までの定積分}
不定積分
$$
    h(z) = \int \frac{1}{|y|} f(z/y) g(y)\,dy
$$

-∞から∞までの定積分
$$
    h(z) = \int_{-\infty}^{\infty} \frac{1}{|y|} f(z/y) g(y)\,dy
$$
\subsubsection{定積分}
\begin{description}
    \item xの積分区間は$x_0 \leq x \leq x_1$
    \item yの積分区間は$y_0 \leq y \leq y_1$
\end{description}
$$
    h(z)=
    \begin{cases}
        \int_{y_0}^{y_1} \frac{1}{|y|} f(z/y) g(y)\,dy  & if\ x_0y \leq z \leq x_1y \ \ \forall y\in \{ y_0 \leq y \leq y_1\},\\
        積分区間を分割するべし                            & else \ if\ x_0y \leq z \leq x_1y \ \ \exists y\in \{ y_0 \leq y \leq y_1\}, \\
        0                                               & otherwise,
    \end{cases}
$$
(一番上の条件、$y_0 \leq y \leq y_1$を満たすすべての$y$が$x_0 \leq z/y \leq x_1$を満たす場合。と言いたい)


\subsection{Z=X/Y}
商の計算。
\subsubsection{不定積分と-∞から∞までの定積分}
不定積分
$$
    h(z) = \int |y| f(yz) g(y)\,dy
$$

-∞から∞までの定積分
$$
    h(z) = \int_{-\infty}^{\infty} |y| f(yz) g(y)\,dy
$$
\subsubsection{定積分}
\begin{description}
    \item xの積分区間は$x_0 \leq x \leq x_1$
    \item yの積分区間は$y_0 \leq y \leq y_1$
\end{description}
$$
    h(z)=
    \begin{cases}
        \int_{y_0}^{y_1} |y| f(yz) g(y)\,dy & if\ x_0 \leq yz \leq x_1 \ \ \forall y\in \{ y_0 \leq y \leq y_1\},\\
        積分区間を分割するべし                & else \ if\ x_0 \leq yz \leq x_1 \ \ \exists y\in \{ y_0 \leq y \leq y_1\}, \\
        0                                   & otherwise,
    \end{cases}
$$
(一番上の条件、$y_0 \leq y \leq y_1$を満たすすべての$y$が$x_0 \leq yz \leq x_1$を満たす場合。と言いたい)

\newpage


\section{確率密度関数が離散値の確率変数X,Yの四則演算}
確率変数X,Yがともに離散値の時を考える章。\\
\\

この章のそれぞれの節における共通事項の一覧。
\begin{description}
    \item $h(z_k) = \sum_{\{ i,j \}_k} f(x_i)g(y_j)$
    \item $i=1,2,3\cdots,N$
    \item $j=1,2,3\cdots,M$
    \item $k=1,2,3\cdots,L$\ \ (Lは$1\leq L \leq N+M$を満たす自然数。)
    \item $x_{i+1}=x_i+d_i$ \ \ \ \ ($d_i>0$)
    \item $y_{j+1}=y_j+d_j$ \ \ \ \ ($d_j>0$)
    \item $z_{k+1}=z_k+d_k$ \ \ \ \ ($d_k>0$)
\end{description}

\subsection{Z=X+Y}
和の
$\{ i,j \}_k = \{ (i,j) | z_k = x_i + y_j \}$

\subsection{Z=X-Y}
差の
$\{ i,j \}_k = \{ (i,j) | z_k = x_i - y_j \}$

\subsection{Z=XY}
積の
$\{ i,j \}_k = \{ (i,j) | z_k = x_i y_j \}$

\subsection{Z=X/Y}
商の
$\{ i,j \}_k = \{ (i,j) | z_k = x_i / y_j \}$



\end{document}