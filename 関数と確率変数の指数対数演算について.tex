\documentclass[a4paper,11pt]{ltjsarticle}
\usepackage{amsmath}
\begin{document}
\tableofcontents
\newpage



\section{共通する前提}
内容の正しさは自信ない。特に数学記号の使い方。\\

これ以降、共通して用いる変数・関数\\
$f(x), g(y), h(z)$:確率密度関数\\
$X, Y, Z$:確率変数\\
$A$:(確率変数ではない)変数および定数\\
$i,j,k,N,M,L$:自然数\\
$X=f(x), Y=g(y), Z=h(z)$\\
特に断りがなければ、X,Y,Zは独立な確率変数\\
特に断りがなければ、x,y,zは独立な変数
\newpage


\section{確率変数XとAの演算}
\subsection{$X^A$}
$Y=X^A$のとき。%ここ考えてるとこ
$g(y) = \frac{f(x/(Ax^{A-1}))}{Ax^{A-1}}$
\ \ \ \ ($y=x^A$)
\subsection{$A^X$}
$Y=A^X$のとき。
$g(y) = $
\ \ \ \ ($y=A^x$)
\subsection{$\log_AX$}
$\log_AX$のとき。
$g(y) = $
\ \ \ \ ($y=\log_Ax$)
\subsection{$\log_XA$}
$\log_XA$のとき。
$g(y) = $
\ \ \ \ ($y=\log_xA$)
\newpage

\section{確率密度関数が連続値の確率変数X,Yの演算$X^Y$}
$Z=X^Y$について。
\subsection{不定積分と$-\infty$から$\infty$までの定積分}
$$
    h(z) = \int \frac{1}{|yz^{1-1/y}|} f(z^{1/y}) g(y) \, dy
$$
$$
    h(z) = \int_{-\infty}^{\infty} \frac{1}{|yz^{1-1/y}|} f(z^{1/y}) g(y) \, dy
$$
\subsection{定積分}
\begin{description}
    \item xの積分区間は$x_0 \leq x \leq x_1$
    \item yの積分区間は$y_0 \leq y \leq y_1$
\end{description}
$$
    h(z)=
    \begin{cases}
        \int_{-\infty}^{\infty} \frac{1}{|yz^{1-1/y}|} f(z^{1/y}) g(y) \, dy    & if\ \exists y,\  y\in \{ y_0 \leq y \leq y_1 \ | \ x_0 \leq z^{1/y} \leq x_1 \}\\
        0                                                                       & otherwise,
    \end{cases}
$$
\newpage

\section{確率密度関数が離散値の確率変数X,Yの演算$X^Y$}


\end{document}